\section{Introducción}
Imagina estar en una tienda y necesitar dar el cambio exacto con la menor cantidad de monedas posible; en una empresa de transporte marítimo que debe seleccionar las mercancías más valiosas sin sobrecargar el barco; o en un repartidor que necesita encontrar la ruta más corta para entregar pedidos a varias casas. Estos escenarios cotidianos ilustran problemas de optimización complejos que pueden ser resueltos eficientemente mediante la programación dinámica. Esta técnica se basa en dividir problemas grandes en subproblemas más pequeños y manejables reutilizando las soluciones de estos subproblemas para construir una solución óptima de manera eficiente.

La programación dinámica no solo es eficaz sino también versátil se puede encontrar aplicaciones en diversas áreas como la logística la gestión de recursos la ingeniería de redes y la toma de decisiones. La complejidad algorítmica de los algoritmos de programación dinámica varía según la estructura del problema y el tamaño de los parámetros de entrada. Esto significa que la eficiencia de estos algoritmos está directamente relacionada con el tamaño del problema haciendo que la programación dinámica sea una herramienta invaluable en la resolución de problemas de gran escala.

En ese sentido el presente informe se centra en tres problemas clásicos resueltos mediante programación dinámica: el problema del cambio de monedas, el problema de la mochila y el problema de las distancias más cortas. A través del análisis y la implementación de algoritmos para estos problemas se demuestra la eficacia de la programación dinámica en la optimización de recursos y la toma de decisiones ofreciendo soluciones prácticas y eficientes en contextos diversos.
