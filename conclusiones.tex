\section{Conclusiones}
La programación dinámica se demuestra como una herramienta poderosa y versátil para resolver problemas de optimización clásicos como el cambio de monedas, el problema de la mochila y el problema de las distancias más cortas. A través de los ejemplos y análisis realizados, se evidencia su aplicabilidad en diferentes contextos, así como la importancia de entender la complejidad algorítmica para mejorar la eficiencia de las soluciones propuestas.
