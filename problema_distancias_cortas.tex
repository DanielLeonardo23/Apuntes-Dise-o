\section{El Problema de las Distancias Más Cortas}
El problema de las distancias más cortas es uno de los temas más estudiados en la teoría de grafos por sus numerosas aplicaciones, incluyendo la ingeniería de redes, la logística, inteligencia artificial y la investigación operativa. El problema trata sobre un grafo que es una estructura compuesta por nodos y aristas. Cada arista es una conexión entre vértices y puede tener un peso asociado que representa la "distancia" o el "costo" de viajar entre dichos nodos. Se desea hallar la ruta de menor coste entre dos nodos.

\subsection{Ejemplo 1 Algoritmo de Floyd-Warshall}

\subsection{Código}
\
\subsection{Resultado}
