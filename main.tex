\documentclass[a4paper,12pt]{article}
\usepackage[utf8]{inputenc}
\usepackage[spanish]{babel}
\usepackage{graphicx}
\usepackage{amsmath}
\usepackage{mathptmx} % Paquete para cambiar a Times New Roman
\usepackage[left=2.54cm, right=2.54cm, top=2.54cm, bottom=2.54cm]{geometry}
\usepackage{tocloft} % Paquete para el control del índice
\usepackage{float} % Para obligar la posición de las figuras
\usepackage{listings} % Para el código fuente
\usepackage{setspace} % Paquete para configurar el interlineado
\usepackage{booktabs} % Para mejorar el aspecto de las tablas
\usepackage{sectsty} % Paquete para configurar el tamaño de los títulos
\usepackage{hyperref}

\lstset{
  basicstyle=\ttfamily\footnotesize,
  breaklines=true,
  frame=single,
  language=Python,
  captionpos=b
}

\doublespacing % Establece el interlineado a doble espacio

% Configuración de los títulos
\allsectionsfont{\normalsize}

% Redefinir el entorno thebibliography para que no incluya el título por defecto
\usepackage{etoolbox}
\patchcmd{\thebibliography}{\section*{\refname}}{}{}{}

\begin{document}

% Portada con formato APA
\begin{titlepage}
    \begin{center}
        \textbf{\large “Año del Bicentenario, de la consolidación de nuestra Independencia, y de la conmemoración de las heroicas batallas de Junín y Ayacucho”}\\
        \vspace{0.5cm}
        \textbf{\large Universidad Nacional Mayor de San Marcos}\\
        \vspace{0.5cm}
        \textbf{\large Universidad del Perú, Decana de América}\\
        \vspace{0.5cm}
        \includegraphics[height=4cm]{unmsm.png}\\ % Ajusta el tamaño de la imagen
        \vspace{5mm}
        \textbf{\large Facultad de Ingeniería de Sistemas e Informática}\\
        \vspace{0.5cm}
        \textbf{\large Escuela Profesional de Ingeniería de Sistemas}\\
        \vspace{0.5cm}
        \textbf{\large Proyecto de fin de curso}\\
        \vspace{0.5cm}
        \textbf{\large Profesor: Luis Guerra Grados}\\
        \vspace{1cm}
    \end{center}

    \begin{center}
        \textbf{\large Integrantes del grupo N° 2:}\\
        \vspace{0.5cm}
        {\large Campos García, Henry Leonardo}\\
        \vspace{0.3cm}
        {\large Escribas Alan, Daniel Leonardo}\\
        \vspace{0.3cm}
        {\large Meléndez Blas, Jhair Roussell}\\
        \vspace{0.3cm}
        {\large Morales Damasco, Cristian Ricardo}\\
        \vspace{0.3cm}
        {\large Ortis Herrera, Fabrizio Peter}\\
        \vspace{1cm}
        {\large 23 / junio / 2024}
    \end{center}
\end{titlepage}


\newpage

% Temporalmente desactivar la entrada del índice para el ToC
\addtocontents{toc}{\protect\setcounter{tocdepth}{-1}}
\renewcommand{\contentsname}{Índice} % Cambiar el nombre del índice
\tableofcontents
\addtocontents{toc}{\protect\setcounter{tocdepth}{2}}
\newpage

\section{Introducción}
Imagina estar en una tienda y necesitar dar el cambio exacto con la menor cantidad de monedas posible; en una empresa de transporte marítimo que debe seleccionar las mercancías más valiosas sin sobrecargar el barco; o en un repartidor que necesita encontrar la ruta más corta para entregar pedidos a varias casas. Estos escenarios cotidianos ilustran problemas de optimización complejos que pueden ser resueltos eficientemente mediante la programación dinámica. Esta técnica se basa en dividir problemas grandes en subproblemas más pequeños y manejables reutilizando las soluciones de estos subproblemas para construir una solución óptima de manera eficiente.

La programación dinámica no solo es eficaz sino también versátil se puede encontrar aplicaciones en diversas áreas como la logística la gestión de recursos la ingeniería de redes y la toma de decisiones. La complejidad algorítmica de los algoritmos de programación dinámica varía según la estructura del problema y el tamaño de los parámetros de entrada. Esto significa que la eficiencia de estos algoritmos está directamente relacionada con el tamaño del problema haciendo que la programación dinámica sea una herramienta invaluable en la resolución de problemas de gran escala.

En ese sentido el presente informe se centra en tres problemas clásicos resueltos mediante programación dinámica: el problema del cambio de monedas, el problema de la mochila y el problema de las distancias más cortas. A través del análisis y la implementación de algoritmos para estos problemas se demuestra la eficacia de la programación dinámica en la optimización de recursos y la toma de decisiones ofreciendo soluciones prácticas y eficientes en contextos diversos.

\newpage
\section{Objetivos}
\subsection{Objetivo General}
Desarrollar y analizar algoritmos eficientes basados en programación dinámica para resolver problemas clásicos de optimización demostrando su aplicabilidad y eficiencia en diferentes contextos.

\subsection{Objetivos Específicos}
\begin{itemize}
    \item Implementar un algoritmo de programación dinámica para resolver el problema del cambio de monedas con restricciones en la cantidad disponible.
    \item Desarrollar un algoritmo basado en programación dinámica para maximizar los ingresos en el problema de la mochila.
    \item Aplicar el algoritmo de Floyd-Warshall para encontrar las rutas de menor coste en un grafo optimizando la logística de entrega.
    \item Analizar la complejidad algorítmica de cada algoritmo implementado destacando los factores que afectan su rendimiento.
\end{itemize}

\newpage
\section{Planteamiento del Problema}
¿De qué manera la programación dinámica resulta ser una solución eficiente y precisa para resolver problemas de optimización como el cambio de la moneda, el problema de la mochila o la selección de rutas óptimas en diversos contextos comerciales?

\newpage

\section{El problema del cambio de monedas}
El problema del cambio de monedas, un desafío clásico en el campo de la optimización combinatoria y las ciencias de la computación, involucra determinar el número mínimo de monedas que se requieren para sumar una cantidad específica de dinero. Este problema se complica aún más cuando las monedas disponibles están limitadas en cantidad. La programación dinámica ofrece una metodología robusta para abordar este tipo de problemas permitiendo una solución eficiente y efectiva.

\subsection{Ejemplo 1}
Un cliente paga una cuenta de 6.20 soles con un billete de 10 soles. El cajero necesita dar el cambio con la menor cantidad de monedas posible. Las denominaciones disponibles son monedas de 1 sol, 50 céntimos, 20 céntimos y 10 céntimos con cantidades limitadas de cada una. El reto es calcular la forma óptima de entregar 3.80 soles de vuelto usando las denominaciones mencionadas. Considere que tiene 2 monedas de 1 sol, seis de 50 céntimos, cinco de 20 céntimos, y diez de 10 céntimos.

\subsubsection{Código}
\begin{lstlisting}
def min_coins_change(total, denominations, counts):
    # Convertimos las cantidades a centimos para manejarlas como enteros
    total_cents = int(total * 100)
    denominations_cents = [int(d * 100) for d in denominations]

    # Inicializamos la tabla de DP con infinito para todas las cantidades
    # excepto para 0 que necesita 0 monedas
    dp = [float('inf')] * (total_cents + 1)
    dp[0] = 0

    # Para rastrear el uso de las monedas correctamente
    used_coins = [[0 for _ in denominations] for _ in range(total_cents + 1)]

    # Procesamos cada tipo de moneda
    for i, coin in enumerate(denominations_cents):
        for j in range(coin, total_cents + 1):
            if dp[j - coin] != float('inf') and counts[i] > used_coins[j - coin][i]:
                if dp[j] > dp[j - coin] + 1:
                    dp[j] = dp[j - coin] + 1
                    used_coins[j] = used_coins[j - coin][:]
                    used_coins[j][i] += 1

    if dp[total_cents] == float('inf'):
        return "No se puede dar el cambio exacto"
    else:
        return dp[total_cents], used_coins[total_cents]

# Denominaciones de las monedas disponibles y sus cantidades
denominations = [1.0, 0.50, 0.20, 0.10]  # en soles
counts = [2, 6, 5, 10]  # cantidad de monedas disponibles

# Cantidad total a devolver
total_change = 3.80  # en soles

min_coins, coin_usage = min_coins_change(total_change, denominations, counts)

print("Numero minimo de monedas:", min_coins)
print("Uso de monedas:", coin_usage)
\end{lstlisting}

\subsubsection{Resultado}
Número mínimo de monedas: 7 \\
Uso de monedas: [2, 3, 1, 1]

\subsubsection{Análisis de complejidad}

\section*{Análisis de Complejidad}



\subsection*{Detalle de Complejidad}

\textbf{1. A: Convertir a centimos y crear lista de denominaciones en centimos.}

El código correspondiente es:
\begin{lstlisting}
total_cents = int(total * 100)
denominations_cents = [int(d * 100) for d in denominations]
\end{lstlisting}

- La conversión del total a centimos: \texttt{total\_cents = int(total * 100)} es una operación constante \( O(1) \).
- La conversión de cada denominación a centimos: \texttt{denominations\_cents = [int(d * 100) for d in denominations]} es una operación que se realiza \( n \) veces, donde \( n \) es el número de denominaciones.

Por lo tanto, la complejidad de esta parte es:
\[ A = O(n) \]
Aquí, \( n \) es el número de denominaciones.

\textbf{2. B: Inicializar la tabla de DP.}

El código correspondiente es:
\begin{lstlisting}
dp = [float('inf')] * (total_cents + 1)
dp[0] = 0
\end{lstlisting}

- Inicializar la tabla \texttt{dp} con infinito: \texttt{dp = [float('inf')] * (total\_cents + 1)} implica recorrer \( m+1 \) posiciones, donde \( m \) es el total en centimos.
- Establecer \texttt{dp[0] = 0} es una operación constante \( O(1) \).

Por lo tanto, la complejidad de esta parte es:
\[ B = O(m+1) \]
Aquí, \( m \) es el total en centimos.

\textbf{3. C: Inicializar \texttt{used\_coins}.}

El código correspondiente es:
\begin{lstlisting}
used_coins = [[0 for _ in denominations] for _ in range(total_cents + 1)]
\end{lstlisting}

- Inicializar la tabla \texttt{used\_coins} implica crear una lista de listas de ceros. La lista externa tiene \( m+1 \) elementos y cada lista interna tiene \( n \) elementos.

Por lo tanto, la complejidad de esta parte es:
\[ C = (m+1) \cdot n \]
\[ C = O((m+1)n) \]

 \textbf{4. D: Procesar cada tipo de moneda.}

El código correspondiente es:
\begin{lstlisting}
for i, coin in enumerate(denominations_cents):
    for j in range(coin, total_cents + 1):
        if dp[j - coin] != float('inf') and counts[i] > used_coins[j - coin][i]:
            if dp[j] > dp[j - coin] + 1:
                dp[j] = dp[j - coin] + 1
                used_coins[j] = used_coins[j - coin][:]
                used_coins[j][i] += 1
\end{lstlisting}

- El bucle externo sobre las denominaciones: \texttt{for i, coin in enumerate(denominations\_cents):} se ejecuta \( n \) veces.
- El bucle interno sobre el rango del total en centimos: \texttt{for j in range(coin, total\_cents + 1):} se ejecuta aproximadamente \( m \) veces para cada denominación.
- Comprobar y actualizar \texttt{dp} y \texttt{used\_coins} es una operación constante \( O(1) \).

Por lo tanto, la complejidad de esta parte es:
\[ D = \sum_{i=1}^{n} \sum_{j=i}^{m} O(1) \]
\[ D = \sum_{i=1}^{n} O(m) \]
\[ D = O(n \cdot m) \]

\textbf{5. E: Condición final.}

El código correspondiente es:
\begin{lstlisting}
if dp[total_cents] == float('inf'):
    return "No se puede dar el cambio exacto"
else:
    return dp[total_cents], used_coins[total_cents]
\end{lstlisting}

- Verificar si se puede dar el cambio exacto: \texttt{if dp[total\_cents] == float('inf')} es una operación constante \( O(1) \).
- Devolver el resultado es una operación constante \( O(1) \).

Por lo tanto, la complejidad de esta parte es:
\[ E = O(1) \]

\textbf{6. F: Impresión de resultados.}

El código correspondiente es:
\begin{lstlisting}
print("Numero minimo de monedas:", min_coins)
print("Uso de monedas:", coin_usage)
\end{lstlisting}

- Imprimir el número mínimo de monedas y el uso de monedas es una operación constante \( O(1) \).

Por lo tanto, la complejidad de esta parte es:
\[ F = O(1) \]

\subsection*{Complejidad Total}
\subsection*{Fórmula }

\[ T(m) = A + B + C + D + E + F \]

Sumando todas las complejidades:
\begin{align*}
T(m) &= O(n) + O(m+1) + O((m+1)n) + O(n \cdot m) + O(1) + O(1) \\
T(m) &= O(n) + O(m) + O((m+1)n) + O(n \cdot m) + O(1) \\
T(m) &= O(n) + O(m) + O(m \cdot n) + O(m \cdot n) + O(1) \\
T(m) &= O(n) + O(m) + O(m \cdot n) \\
T(m) &= O(m \cdot n) \\
\end{align*}

\subsection*{Variables}

\begin{itemize}
    \item \( n \) = número de denominaciones de monedas.
    \item \( m \) = monto total en centimos.
\end{itemize}

\subsection{Ejemplo 2}
En primer lugar, debemos pensar cómo plantear el problema de forma incremental. Consideramos el tipo de moneda de mayor valor, \(X_N\). Si \(X_N > C\) entonces la descartamos y pasamos a considerar monedas de menor valor. Si \(X_N \leq C\) tenemos dos opciones: o tomar una moneda de tipo \(X_N\) y completar la cantidad restante \(C - X_N\) con otras monedas, o no tomar ninguna moneda de tipo \(X_N\) y completar la cantidad \(C\) con monedas de menor valor. De las dos opciones, nos quedamos con la que requiera un número menor de monedas. El problema lo podemos expresar de la siguiente forma cuando consideramos \(N\) tipos de monedas:

\[
Cambio(N, C) = 
\begin{cases} 
cambio(N-1, C) & \text{si } X_N > C \\
\min\{cambio(N-1, C), cambio(N, C - X_N) + 1\} & \text{si } X_N \leq C 
\end{cases}
\]

Podemos razonar análogamente para monedas de valores \(k\) menores que \(N\) y para cantidades \(C'\) menores que \(C\):

\[
Cambio(k, C') = 
\begin{cases} 
cambio(k-1, C') & \text{si } X_k > C' \\
\min\{cambio(k-1, C'), cambio(N, C' - X_k) + 1\} & \text{si } X_k \leq C' 
\end{cases}
\]

Llegamos a los casos base de la recurrencia cuando completamos la cantidad \(C\):

\[
cambio(k, 0) = 0 \quad \text{si } 0 \leq k \leq n
\]

o cuando ya no quedan más tipos de monedas por considerar, pero aún no se ha completado la cantidad \(C\):

\[
cambio(0, C') = \infty \quad \text{si } 0 < C' \leq C
\]

Podemos construir una tabla para almacenar los resultados parciales que tenga una fila para cada tipo de moneda y una columna para cada cantidad posible entre 1 y \(C\). Cada posición \(t[i,j]\) será el número mínimo de monedas necesario para dar una cantidad \(j\) con \(0 \leq j \leq C\) utilizando solo monedas de los tipos entre 1 e \(i\), con \(0 \leq i \leq n\). La solución al problema será, por tanto, el contenido de la casilla \(t[N, C]\). Para construir la tabla, empezamos rellenando los casos base \(t[i, 0] = 0\), para todo \(i\) con \(0 \leq i \leq n\). A continuación, podemos rellenar la tabla bien por filas de izquierda a derecha, o bien por columnas de arriba a abajo.

Siguiendo el método de la \textbf{programación dinámica}, se rellenará una tabla con las filas correspondientes a cada valor para las monedas y las columnas con valores desde el 1 hasta el \(N\) (12 en este caso). Cada posición \((i, j)\) de la tabla nos indica el número mínimo de monedas requeridas para devolver la cantidad \(j\) con monedas con valor menor o igual al de \(i\):

\[
\begin{array}{c|ccccccccccccc}
 & 0 & 1 & 2 & 3 & 4 & 5 & 6 & 7 & 8 & 9 & 10 & 11 & 12 \\ \hline
x = 1 & 0 & 1 & 2 & 3 & 4 & 5 & 6 & 7 & 8 & 9 & 10 & 11 & 12 \\
x = 6 & 0 & 1 & 2 & 3 & 4 & 5 & 1 & 2 & 3 & 4 & 5 & 2 & 2 \\
x = 10 & 0 & 1 & 2 & 3 & 4 & 5 & 1 & 2 & 3 & 4 & 5 & 2 & 2 \\
\end{array}
\]

En el caso anterior, hay que pagar 12 con monedas de 1, 6, 10. Supongamos que queremos pagar 12 o pagamos con 12 monedas de 1, o 2 monedas de 6, o con una de 10 y 2 de 1. Como la mejor opción es la de las monedas de 6, me quedo con esa y es la que marco en la tabla. Obsérvese que a pesar de que con monedas de 10 me haría falta 3 monedas, marco solo 2 porque es la mejor opción.

\subsubsection{Código}
\begin{lstlisting}
def min(a, b):
    return a if a < b else b

class Cambio:
    def __init__(self, cantidad, monedas):
        self.cantidad = cantidad
        self.vectorMonedas = monedas
        self.matrizCambio = self.calcularMonedas(cantidad, monedas)
        self.vectorSeleccion = self.seleccionarMonedas(cantidad, monedas, self.matrizCambio)
    
    def getVectorSeleccion(self):
        return self.vectorSeleccion
    
    def calcularMonedas(self, cantidad, monedas):
        matrizCambio = [[0 if j == 0 else 99 for j in range(cantidad + 1)] for i in range(len(monedas) + 1)]
        
        for i in range(1, len(monedas) + 1):
            for j in range(1, cantidad + 1):
                if j < monedas[i - 1]:
                    matrizCambio[i][j] = matrizCambio[i - 1][j]
                else:
                    matrizCambio[i][j] = min(matrizCambio[i - 1][j], matrizCambio[i][j - monedas[i - 1]] + 1)
        
        return matrizCambio
    
    def seleccionarMonedas(self, c, monedas, tabla):
        i, j = len(monedas), c
        seleccion = [0] * len(monedas)
        
        while j > 0:
            if i > 1 and tabla[i][j] == tabla[i - 1][j]:
                i -= 1
            else:
                seleccion[i - 1] += 1
                j -= monedas[i - 1]
        
        return seleccion

# Ejemplo de uso
cantidad = 11
monedas = [1, 5, 6, 9]

# Crear una instancia de Cambio con los valores del ejemplo
cambio_instance = Cambio(cantidad, monedas)

# Obtener el resultado de vectorSeleccion
vector_seleccion = cambio_instance.getVectorSeleccion()
print("Resultado:", vector_seleccion)
\end{lstlisting}

\subsubsection{Resultado}
Para el ejemplo dado con una cantidad de 11 y monedas de denominaciones \([1, 5, 6, 9]\), el resultado del algoritmo es el siguiente:
\begin{itemize}
    \item Utiliza 1 moneda de 5
    \item Utiliza 1 moneda de 6
    \item No utiliza monedas de 1
    \item No utiliza monedas de 9
\end{itemize}
Esto se refleja en el vector de selección: \([0, 1, 1, 0]\).

\subsubsection{Análisis de complejidad}
Inicialización de Variables y Estructuras:
\begin{itemize}
    \item \texttt{calcularMonedas(int cantidad, int[] monedas)}:
    \begin{itemize}
        \item Se inicializa una matriz de tamaño \((monedas.length + 1) \times (cantidad + 1)\).
        \item Complejidad: \(O(m \cdot n)\).
    \end{itemize}
    \item Llenado de la Matriz con Valores Iniciales:
    \begin{itemize}
        \item Complejidad: \(O(m + n)\).
    \end{itemize}
    \item Cálculo de la Matriz de Cambio:
    \begin{itemize}
        \item Complejidad: \(O(m \cdot n)\).
    \end{itemize}
    \item Selección de Monedas:
    \begin{itemize}
        \item Complejidad: \(O(m + n)\).
    \end{itemize}
    \item Función de Utilidad \texttt{min(int a, int b)}:
    \begin{itemize}
        \item Función llamada cálculo de la matriz.
        \item Complejidad: \(O(1)\) por cada llamada.
    \end{itemize}
\end{itemize}

La complejidad total del algoritmo es: \(O(m \cdot n)\). El algoritmo tiene una eficiencia polinómica en términos del número de denominaciones y la cantidad total, lo cual es razonable para muchos problemas prácticos de cambio de monedas. La matriz matrizCambio utiliza \(O(m \cdot n)\) memoria, lo cual puede ser una limitación si \(m\) o \(n\) son muy grandes.



\newpage
\section{El Problema de la Mochila}
El problema de la mochila es un clásico en el ámbito de la optimización combinatoria y la teoría de algoritmos. Se trata de seleccionar un subconjunto de artículos, cada uno con un peso y un valor, de manera que se maximice el valor total sin exceder la capacidad de peso permitida. Este problema es fundamental en campos como la logística, la gestión de recursos y la toma de decisiones en la ingeniería. La programación dinámica y las técnicas de aproximación ofrecen métodos eficientes para resolver este problema.

\subsection{Ejemplo 1}
Un naviero tiene un buque carguero con capacidad de hasta 500 toneladas. El carguero transporta contenedores de diferentes pesos para una determinada ruta. En la ruta actual el carguero puede transportar algunos de los siguientes contenedores:


\begin{table}[H]
	\centering
	\begin{tabular}{cccccc}
		\toprule
		\textbf{Contenedor} & 1 & 2 & 3 & 4 & 5 \\
		\midrule
		Peso en (cientos de toneladas) & 1 & 2 & 1 & 3 & 4 \\
		Valor (miles de dólares) & 3 & 5 & 4 & 6 & 7 \\
		\bottomrule
	\end{tabular}
	\caption{Datos del problema de la mochila. Elaboración propia.}
	\label{tab:datos_contenedores}
\end{table}


El analista de la empresa del armador desea determinar el envío (conjunto de contenedores) que maximiza el valor de la carga transportada.

\subsubsection{Código}
\begin{lstlisting}
	def algoritmo_mochila(pesos, valores, capacidad):
	n = len(pesos)  
	matriz = []
	for i in range(n + 1):
	fila = []
	for j in range(capacidad + 1):
	fila.append(0)  
	matriz.append(fila)  
	
	
	for i in range(1, n + 1):  
	for w in range(1, capacidad + 1):  
	if pesos[i - 1] <= w:  
	matriz[i][w] = max(matriz[i - 1][w], valores[i - 1] + matriz[i - 1][w - pesos[i - 1]])  
	else:
	matriz[i][w] = matriz[i - 1][w]  
	
	w = capacidad  
	contenedores_seleccionados = []  
	for i in range(n, 0, -1):  
	if matriz[i][w] != matriz[i - 1][w]:  
	contenedores_seleccionados.append(i)
	w -= pesos[i - 1]
	
	
	valor_maximo = matriz[n][capacidad]
	contenedores_seleccionados.reverse()  
	return valor_maximo, contenedores_seleccionados
	
\end{lstlisting}

\subsubsection{Resultado}
\begin{figure}[H]
	\centering
	\includegraphics[width=0.8\textwidth]{resultado_mochila_ejem1.png}
	\caption{imagen del resultado del codigo Python.}
	\label{fig:resultado}
\end{figure}

\subsubsection{Análisis de complejidad}
\begin{figure}[H]
	\centering
	\includegraphics[width=0.8\textwidth]{complejidad_mochila_ejem1.png}
	\caption{Analisis del codigo.}
	\label{fig:complejidad1}
\end{figure}

\begin{figure}[H]
	\centering
	\includegraphics[width=0.8\textwidth]{complejidad_mochila_ejem1_2.png}
	\caption{analisis del codigo.}
	\label{fig:complejidad2}
\end{figure}

Como podemos observar , la complejidad del algoritmo esta en función de la variable C que es la capacidad, es decir, si la capacidad es demasiado grande  , la complejidad será mayor , caso contrario la complejidad será mínima.

\subsection{Ejemplo 2}
Una empresa de transporte marítimo de Mercancías posee un barco con una bodega cuya capacidad es de 250 cm³. Desea transportar cuatro bienes de los que se dispone su volumen y su valor monetario. En la siguiente tabla se muestra dicha información:


\begin{table}[h!]
	\centering
	\begin{tabular}{cccc}
		\toprule
		\textbf{BIENES} & \textbf{VOLUMEN} & & \textbf{Ingresos (\$)} \\
		& \textbf{(cm\(^3\)/Tm)} & & \\
		\midrule
		1 & 70 & & 1250 \\
		2 & 50 & & 900 \\
		3 & 60 & & 1000 \\
		4 & 75 & & 1200 \\
		\bottomrule
	\end{tabular}
	\caption{Datos del problema de la mochila. Elaboración propia.}
	\label{tab:datos_problema}
\end{table}

Se trata de determinar los bienes que se debe transportar en cada bodega de forma que el ingreso sea máximo. 

\subsubsection{Código}
\begin{lstlisting}[language=Python]
	def mochila(capacidad, volumenes, ingresos, n):
	# Crear una matriz para almacenar los ingresos maximos posibles
	dp = [[0 for x in range(capacidad + 1)] for x in range(n + 1)]
	
	# Llenar la matriz dp de manera ascendente
	for i in range(n + 1):
	for w in range(capacidad + 1):
	if i == 0 or w == 0:
	dp[i][w] = 0
	elif volumenes[i - 1] <= w:
	dp[i][w] = max(ingresos[i - 1] + dp[i - 1][w - volumenes[i - 1]], dp[i - 1][w])
	else:
	dp[i][w] = dp[i - 1][w]
	
	# Recuperar los elementos seleccionados
	res = dp[n][capacidad]
	w = capacidad
	bienes_seleccionados = []
	
	for i in range(n, 0, -1):
	if res <= 0:
	break
	if res == dp[i - 1][w]:
	continue
	else:
	bienes_seleccionados.append(i - 1)
	res -= ingresos[i - 1]
	w -= volumenes[i - 1]
	
	# Crear una lista que indique si cada bien se lleva (1) o no (0)
	seleccion = [0] * n
	for i in bienes_seleccionados:
	seleccion[i] = 1
	
	return dp[n][capacidad], bienes_seleccionados, seleccion
	
	# Datos del problema
	volumenes = [70, 50, 60, 75]
	ingresos = [1250, 900, 1000, 1200]
	capacidad = 250
	n = len(volumenes)
	
	# Resolver el problema
	ingreso_maximo, bienes_seleccionados, seleccion = mochila(capacidad, volumenes, ingresos, n)
	
	print(f"El ingreso maximo es: ${ingreso_maximo}")
	print("Bienes seleccionados (indices):", bienes_seleccionados)
	print("Seleccion de bienes (1 = seleccionado, 0 = no seleccionado):")
	for i in range(n):
	print(f"Bien {i + 1}: {seleccion[i]}")
\end{lstlisting}

Código extraído de Untitled0.ipynb


\subsubsection{Resultado}
\begin{figure}[H]
	\centering
	\includegraphics[width=0.8\textwidth]{resultado_mochila_ejem2.png}
	\caption{imagen del resultado del codigo Python.}
	\label{fig:resultado}
\end{figure}

\subsubsection{Análisis de complejidad}
\begin{figure}[H]
	\centering
	\includegraphics[width=0.8\textwidth]{complejidad_mochila_ejem2.png}
	\caption{Analisis del codigo.}
	\label{fig:complejidad1}
\end{figure}

\begin{figure}[H]
	\centering
	\includegraphics[width=0.8\textwidth]{complejidad_mochila_ejem2_2.png}
	\caption{analisis del codigo.}
	\label{fig:complejidad2}
\end{figure}

Como se acaba de apreciar, el presente algoritmo tiene una complejidad aproximada de O(n x K) o equivalentemente  n x capacidad, lo cual supone eficiencia en función al dato de entrada de “capacidad”, mientras más sea esto, más complejo se vuelve.

\newpage
\section{El Problema de las Distancias Más Cortas}
El problema de las distancias más cortas es uno de los temas más estudiados en la teoría de grafos por sus numerosas aplicaciones, incluyendo la ingeniería de redes, la logística, inteligencia artificial y la investigación operativa. El problema trata sobre un grafo que es una estructura compuesta por nodos y aristas. Cada arista es una conexión entre vértices y puede tener un peso asociado que representa la "distancia" o el "costo" de viajar entre dichos nodos. Se desea hallar la ruta de menor coste entre dos nodos.

\subsection{Ejemplo 1 Algoritmo de Floyd-Warshall}

\subsection{Código}
\
\subsection{Resultado}

\newpage
\section{Conclusiones}
La programación dinámica se demuestra como una herramienta poderosa y versátil para resolver problemas de optimización clásicos como el cambio de monedas, el problema de la mochila y el problema de las distancias más cortas. A través de los ejemplos y análisis realizados, se evidencia su aplicabilidad en diferentes contextos, así como la importancia de entender la complejidad algorítmica para mejorar la eficiencia de las soluciones propuestas.

\newpage

% Añadir manualmente "Referencia" al índice
\section*{Referencia}
\addcontentsline{toc}{section}{Referencia}

% Incluir las referencias desde otro archivo sin duplicar el título
\begin{thebibliography}{9}

    \bibitem{jcsis2013}
    jcsis. (2013, 22 de diciembre). Programación dinámica: algoritmo cambio monedas con tipo de monedas a devolver. \textit{jcsis Blog}. Recuperado de \url{https://jcsis.wordpress.com/2013/12/22/programacion-dinamica-algoritmo-cambio-monedas/}
        
    \bibitem{pdfcoffee2024}
    The Knapsack Problem. \textit{pdfcoffee.com}. (Fecha y autores no disponibles). Recuperado de \url{https://pdfcoffee.com/knapsack-problem-10-pdf-free.html}
        
    \bibitem{mendoza2021}
    Mendoza, I., \& Ruedas, L. (2021, 31 de enero). Algoritmo de Floyd-Warshall: Análisis e implementación. \textit{Medium}. Recuperado el 6 de mayo de 2024, de \url{https://medium.com/algoritmo-floyd-warshall/algoritmo-de-floyd-warshall-e1fd1a900d8}
        
    \bibitem{kleinberg2013}
    Kleinberg, J., \& Tardos, E. (2013). \textit{Algorithm Design} (1ª ed.). Pearson. Recuperado de \url{https://www.perlego.com/book/810837/algorithm-design-pdf} (Trabajo original publicado en 2013)
        
    \bibitem{cormen1990}
    Cormen, T. H., Leiserson, C. E., \& Rivest, R. L. (1990). \textit{Introduction to algorithms}. Cambridge, MA: MIT Press.
        
    \bibitem{ortega2017}
    Kevin, O. [Kevin Ortega]. (2017, 2 de febrero). Algoritmo Bellman-Ford: Definición y ejemplo del algoritmo Bellman-Ford empleado por el protocolo de enrutamiento de rutas RIP. \textit{Ing Raúl Lozada Yanez} [Video]. YouTube. Recuperado el 11 de mayo de 2024, de \url{https://www.youtube.com/watch?v=CIz_8J6IuM0}

\end{thebibliography}

\newpage

\end{document}
